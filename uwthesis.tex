%  ========================================================================
%  Copyright (c) 1985 The University of Washington
%
%  Licensed under the Apache License, Version 2.0 (the "License");
%  you may not use this file except in compliance with the License.
%  You may obtain a copy of the License at
%
%      http://www.apache.org/licenses/LICENSE-2.0
%
%  Unless required by applicable law or agreed to in writing, software
%  distributed under the License is distributed on an "AS IS" BASIS,
%  WITHOUT WARRANTIES OR CONDITIONS OF ANY KIND, either express or implied.
%  See the License for the specific language governing permissions and
%  limitations under the License.
%  ========================================================================
%

% Documentation for University of Washington thesis LaTeX document class
% by Jim Fox
% fox@washington.edu
%
%    Revised 2020/02/24, added \caption()[]{} option.  No ToC.
%
%    Revised for version 2015/03/03 of uwthesis.cls
%    Revised, 2016/11/22, for cleanup of sample copyright and title pages
%
%    This document is contained in a single file ONLY because
%    I wanted to be able to distribute it easily.  A real thesis ought
%    to be contained on many files (e.g., one for each chapter, at least).
%
%    To help you identify the files and sections in this large file
%    I use the string '==========' to identify new files.
%
%    To help you ignore the unusual things I do with this sample document
%    I try to use the notation
%       
%    % --- sample stuff only -----
%    special stuff for my document, but you don't need it in your thesis
%    % --- end-of-sample-stuff ---


%    Printed in twoside style now that that's allowed
%
 
\documentclass [11pt, proquest] {uwthesis}[2020/02/24]
 
%
% The following line would print the thesis in a postscript font 

% \usepackage{natbib}
% \def\bibpreamble{\protect\addcontentsline{toc}{chapter}{Bibliography}}

\setcounter{tocdepth}{1}  % Print the chapter and sections to the toc
 

% ==========   Local defs and mods
%

% --- sample stuff only -----
% These format the sample code in this document

\usepackage{alltt}  % 
\newenvironment{demo}
  {\begin{alltt}\leftskip3em
     \def\\{\ttfamily\char`\\}%
     \def\{{\ttfamily\char`\{}%
     \def\}{\ttfamily\char`\}}}
  {\end{alltt}}
 
% metafont font.  If logo not available, use the second form
%
% \font\mffont=logosl10 scaled\magstep1
\let\mffont=\sf
% --- end-of-sample-stuff ---
 



\begin{document}
 
% ==========   Preliminary pages
%
% ( revised 2012 for electronic submission )
%

\prelimpages
 
%
% ----- copyright and title pages
%
\Title{a dissertation}
\Author{Julie L. Newcomb}
\Year{2021}
\Program{Computer Science and Engineering}

\Chair{Name of Chairperson}{Title of Chair}{Department of Chair}
\Signature{First committee member}
\Signature{Next committee member}
\Signature{etc}

\copyrightpage

\titlepage  

 
%
% ----- signature and quoteslip are gone
%

%
% ----- abstract
%


\setcounter{page}{-1}
\abstract{%
This sample dissertation is an aid to students who are attempting
to format their theses with \LaTeX, a sophisticated
text formatter widely used by mathematicians and scientists everywhere.
 
\begin{itemize}
\item It describes the use of a specialized
macro package developed specifically for thesis production
at the University.
The macros customize \LaTeX\ for the correct thesis style,
allowing the student to concentrate on the substance of
his or her text.%
\footnote{See Appendix A to obtain the source to this
 thesis and the class file.}
\item It demonstrates the solutions to a variety of
formatting challenges found in thesis production.
\item It serves as a template for a real dissertation.
\end{itemize}
}
 
%
% ----- contents & etc.
%
\tableofcontents
\listoffigures
%\listoftables  % I have no tables
 
%
% ----- glossary 
%
\chapter*{Glossary}      % starred form omits the `chapter x'
\addcontentsline{toc}{chapter}{Glossary}
\thispagestyle{plain}
%
\begin{glossary}
\item[argument] replacement text which customizes a \LaTeX\ macro for
each particular usage.
\item[back-up] a copy of a file to be used when catastrophe strikes
the original.  People who make no back-ups deserve
no sympathy.
\item[control sequence] the normal form of a command to \LaTeX.
\item[delimiter] something, often a character, that indicates
the beginning and ending of an argument.
More generally, a delimiter is a field separator.
\item[document class] a file of macros that tailors \LaTeX\ for
a particular document.  The macros described by this thesis
constitute a document class.
\item[document option] a macro or file of macros
that further modifies \LaTeX\ for
a particular document.  The option {\tt[chapternotes]}
constitutes a document option.
\item[figure] illustrated material, including graphs,
diagrams, drawings and photographs.
\item[font] a character set (the alphabet plus digits
and special symbols) of a particular size and style.  A couple of fonts
used in this thesis are twelve point roman and {\sl twelve point roman
slanted}.
\item[footnote] a note placed at the bottom of a page, end of a chapter,
or end of a thesis that comments on or cites a reference
for a designated part of the text.
\item[formatter] (as opposed to a word-processor) arranges printed
material according to instructions embedded in the text.
A word-processor, on the other hand, is normally controlled
by keyboard strokes that move text about on a display.
\item[\LaTeX] simply the ultimate in computerized typesetting.
\item[macro]  a complex control sequence composed of 
other control sequences.
\item[pica] an archaic unit of length.  One pica is twelve points and
six picas is about an inch.
\item[point] a unit of length.  72.27 points equals one inch.
\item[roman]  a conventional printing typestyle using serifs.
the decorations on the ends of letter strokes.
This thesis is set in roman type.
\item[rule] a straight printed line; e.g., \hrulefill.
\item[serif] the decoration at the ends of letter strokes.
\item[table] information placed in a columnar arrangement.
\item[thesis] either a master's thesis or a doctoral dissertation.
This document also refers to itself as a thesis, although it
really is not one.
 
\end{glossary}
 
%
% ----- acknowledgments
%
\acknowledgments{% \vskip2pc
  % {\narrower\noindent
  The author wishes to express sincere appreciation to
  University of Washington, where he has had the opportunity
  to work with the \TeX\ formatting system,
  and to the author of \TeX, Donald Knuth, {\it il miglior fabbro}.
  % \par}
}

%
% ----- dedication
%
\dedication{\begin{center}to my dear wife, Joanna\end{center}}

%
% end of the preliminary pages
 
 
 
%
% ==========      Text pages
%

\textpages

% ========== chapters
\chapter{Introduction}
\label{sec:intro}

Term rewriting systems are a useful tool in the construction of compilers. They are a succinct and efficient way of executing program reasoning tasks, and can give extremely fast performance. 

However, term rewriting systems can be challenging both to develop and to maintain. As the path of execution for inputs can vary greatly, it can be difficult for developers to reason about how the various rules in the system interact as a whole. Adding, altering, or deleting a rule can have unforeseen consequences. Even if a TRS contains unsound rules, it may give correct results on all but a few inputs, and when the error is detected it can be difficult to track down which rules is responsible. A slight change to the ruleset can make it possible for rules form cycles, causing the system to fail to terminate on some inputs. 

The task of authoring these systems is even more challenging when the language of input expressions is in an undecidable theory. Since a term rewriting system in such a language cannot be complete, it is hard to know if a given system is optimal or if some change would improve its performance, degrade it, or perhaps make no difference at all.

In this work, we propose to show how authoring and maintaining term rewriting systems can be made easier through the use of formal methods and program synthesis. We will demonstrate how these techniques can support all degrees of human involvement: from verifying and strengthening existing term rewriting systems, to bootstrapping term rewriting systems from a specification and a set of seed axioms, to synthesizing a new ruleset entirely from scratch.

In this work we make the following contributions:

\begin{itemize}
    \item \emph{We show formal methods and program synthesis can usefully assist the authors of term rewriting systems.} Formal proofs of soundness and guarantees of termination can find bugs in large, widely-deployed term rewriting systems, and automatic program synthesis can be used to find new rewrite rules to augment these systems.
    \item \emph{We present an effective means of encoding a specification for the synthesis of term rewriting systems.} Besides providing a proof of termination, we claim that reduction orders can serve as a useful formalization of the intent behind a term rewriting system, and can be used as a specification when synthesizing rewrite rules.
\end{itemize}

We evaluate our work in two case studies. In the first, we investigate the first claim by using formal methods to enhance a pivotal term rewriting system within the Halide compiler called the simplifier. We found that we were able to identify bugs, prove the absence of future errors, and increase the rewriting power of the system. We review our evaluation of the first claim in section~\ref{sec:prior}. 

In the second case study, we take a look at a smaller and less mature term rewriting system in Halide called the variable solver. Previously, we synthesized new rules for the simplifier largely guided by its large existing ruleset. Here, we plan to evaluate our proposed means of writing specifications for term rewriting system by synthesizing a ruleset entirely from scratch. We lay out our means of writing specification and our evaluation of the synthesized rulesets in section~\ref{intro:varsolver}.

\section{Maintaining an existing term rewriting system}
\label{sec:prior}

The Halide compiler contains a hand-authored term rewriting system commonly called the simplifier. It is used in many cases within the compiler for making expressions shorter and in a form better suited to downstream uses. Sometimes the simplifier is used as a prover: the truth value of an expression is checked by rewriting it with the simplifier to see if it will be rewritten to the constant true. At the time we began this work, the simplifier was fairly mature: it had been deployed in production for over a year, was both unit tested and fuzz tested, and comprised almost a thousand rules.

Is it necessary to apply formal methods to the authoring of term rewriting systems? In previous work, we showed that applying formal methods can identify and remedy real issues in the simplifier TRS. Conversely, we observe that implementing expression-transforming code as a term rewriting system is useful precisely because it allows easy integration with formal methods. We also show that we can improve this term rewriting system by synthesizing new rewrite rules automatically and that this synthesis process can be used in the compiler's development process. We first provide a proof of soundness and of termination, and then detail the rule synthesis process below. This work has been published at OOPSLA 2020~\cite{newcomb2020verifying}.

\subsection{Proof of soundness}
First, we formally verify that each rule in the Halide simplifier ruleset is semantics-preserving, demonstrating that proofs of soundness are possible despite the lack of a 
decision procedure for our theory. We do this by modeling the semantics of the Halide expression language in SMT2. We then implemented a pretty-printer to translate each rule in the simplifier ruleset to an SMT query to be checked by the solver Z3~\cite{de2008z3}. About 12\% of the ruleset could not be verified by Z3, and we proved those rules correct by hand using the proof assistant Coq~\cite{Coq19}. Even though the code had been deployed for over a year and had been fuzz-tested, we found that four of the existing rules were incorrect and submitted patches. In constructing the Coq proofs, we also noticed that 17 rules had predicates that were overly conservative, and submitted patches for the relaxed predicates as well.

While this project was ongoing, the Halide semantics for division was changed: division by zero was no longer undefined behavior, but now returned zero. It was simple for us to amend our modeled semantics and rerun rule verification in Z3. Again about 12\% of the ruleset had to be hand-proven using Coq, but we were able to leverage many of the existing proofs from our previous verification. Our verification identified 44 rules which were not correct under the new semantics and 37 rules whose predicates could be relaxed under the semantics change. This shows the value of our formal methods infrastructure, which allowed Halide developers to push a fairly major change with a higher degree of confidence in the soundness of the simplifier than could have been achieved with either manual testing or fuzzing.

\subsection{Proof of termination}
The Halide simplifier algorithm successively applies rewrite rules until the resulting expression can no longer be rewritten. Thus, if some sequence of rewrites to some input expression can form a cycle, the simplifier algorithm will not terminate. These non-termination errors have been observed in the past (resulting in the compiler throwing a stack overflow error and crashing). Without a specific input expression on which to reproduce a cycle, it is very difficult to examine a set of around a thousand rules and find a subset on which a cycle could occur; even once a cycle has been identified, it is difficult to know the best way to repair it. The ruleset was thus very brittle; deleting, altering, or reordering existing rules has caused new non-termination errors in the past as well.

A term rewriting system can be proven to terminate using a formalism called a \emph{reduction order}~\cite{baader1999term}. A reduction order is an order over a language of terms; if for every rule in a term rewriting system, we can show that the rule's left-hand side is strictly greater than the right-hand side in this order, then we know that there can be no set of rules that can form a cycle, and thus the ruleset must always terminate. A reduction order is distinguished by a few special properties to ensure that these ordering holds no matter how input expressions are matched to the left-hand side term, as well as when a rule is used to rewrite a subterm inside of a larger input expression; see ~\cite{newcomb2020verifying} for full details.

We devised a reduction order that fit as many of the existing simplifier rules as possible. Eight rules could not be fit to our ordering, and we submitted patches to either delete or modify them. Once this was done, not only was a class of bugs eliminated, but the ruleset could now be safely modified without fear of introducing new non-termination behavior. It was also now safe to add any new rule so long as the rule conformed to the reduction order. 

We observe that this reduction order serves to encode the meaning of simplification in the context of the Halide TRS
by formalizing what it means for a rule to usefully modify an expression.  While
many notions of ``simpler'' are possible, we encode the specific criteria for Halide
expressions by defining an \emph{ordering} over the left-hand and right-hand terms of 
the rules in the TRS that captures the intentions behind the rewrites. This ordering 
means that every local change caused by the application of a rewrite rule moves the 
expression in some useful direction. By composing several orders lexicographically, 
we can express many different intuitions about what makes an expression simpler in a
defined priority: we may want to remove as many vector operations as possible, 
then reduce the overall size of the expression, and so on. Maintaining this order as 
an invariant over any future rules means any additions to the ruleset will not undo 
progress made by existing rules.

\subsection{Strengthening the ruleset}
We can empirically observe that the simplifier ruleset is not sufficiently powerful to deal with all expressions it may be called upon, by instrumenting the compiler and logging any expressions that the simplifier cannot further solve. In the past, ruleset authors might look at these failed expressions and write rules to address them by hand. In ~\cite{newcomb2020verifying}, we automated this progress by synthesizing these rules instead.

However, even
given the constraints imposed by the termination order, the space of equivalences 
in the Halide expression language is infinitely large. How do we choose which  
rules to add? We observe that there is some bias on the distribution of expressions 
seen by the compiler on realistic inputs over the full expression space. We take advantage 
of this bias by gathering expressions from realistic compilations on which the current 
TRS is ``stuck'' and can make no further progress. We synthesize rules through a pipeline that takes these expression as input and uses CEGIS loops to synthesize an equivalent right-hand side and (if necessary) a predicate guard that indicates when it is safe to apply 
the rule. Our pipeline produces more general rules by mining input expressions for larger patterns and by replacing constants with fresh variables. We also find variants on synthesized rules by applying associativity and commutitativity laws to their left-hand sides. We choose candidate left-hand sides 
for rules from this corpus and synthesize equivalent right-hand terms that obey the 
termination order. In our experiments, we found that our synthesis pipeline could produce patches to ruleset bugs as good or better than those that were authored by hand. We also observed Halide developers integrating the rule synthesizer into their workflow. Our synthesis procedure is also capable of finding large numbers of useful rules without
human oversight; although the existing compiler is mature 
and well-tuned for our suite of benchmarks, we show some performance 
gains without increases in compilation time when our newly synthesized suite 
of rules is added to the TRS. 

 

\section{Synthesizing a new term rewriting system}
\label{intro:varsolver}

For the Halide simplifier, we synthesized individual rules to augment an already mature term rewriting system. Next, we demonstrate the synthesis of term rewriting systems completely from scratch. We do this using the Halide \emph{variable solver} as a case study. The variable solver takes a Halide expression and a variable found within that expression (the \emph{target variable}) and attempts to eliminate as many instances of that variable as it can and isolate the variable to one side of the expression as best it can. (If the expression is an equality, this would consitute `solving' the expression for the chosen variable.) 

An idealized version of our proposed synthesis procedure is laid out in algorithm~\ref{algo:synthesis}. This procedure requires a goal function, which returns true if a term is in some desired form and false otherwise; a semantic equivalence relation; and a reduction order over terms, as well of a set of training expressions $E$, which should be representative of the types of expressions the term rewriting system will take as input and a set of test expressions $S$ on which to evaluate the behavior of the term rewriting system.

The key input to the synthesis procedure is the reduction order over terms, which serves as a specification for the synthesis of rules. The reduction order is a means of formalizing either the intent of the term rewriting system\textemdash the form into which the TRS should rewrite expressions\textemdash or the strategy by which the TRS will rewrite expressions to be closer to the desired form. While TRS authors are likely to have good intuition as to what rewriting strategy their system should employ, choosing the best reduction order for the task is still not a trivial exercise. Our synthesis process gives users the chance to experiment and determine which reduction order serves their purpose best; rather than hand-turning rules, they can synthesize a ruleset entirely from scratch for a given reduction order and evaluate it directly.

We use the Halide variable solver as a case study to demonstrate how synthesis can be used to evaluate reduction order specifications. We lay out the refined synthesis algorithm, with some discussion of how best to search the space of candidate rules, and show how reduction orders can be encoded efficiently as metasketches. We evaluate our claims by synthesizing two rulesets from two different reduction orders in a fully automated way and evaluate their performance against the existing handwritten solver on a suite of benchmarks.


 
% ========== Chapter 1
 
\chapter {Introduction}
 
The utility of a clean, professionally prepared thesis is well
documented%
\footnote{See, for example,
  W.~Shakespeare\cite{Hamlet} for a recent discussion.}
and, even if you never intend to actually print your thesis,
you still ought to format it as if that were your intention.
 
\TeX\ facilitates that. It is a flexible,
complete and professional typesetting system.
It will produce {\bf pdf} output as required by the Graduate School.

\section{The Purpose of This Sample Thesis}
 
This sample is both a demonstration of the quality and
propriety of a \LaTeX formatted thesis and  
documentation for its preparation.
It has made extensive use of a custom class file
developed specifically for this purpose
at the University of Washington.  Chapter~II discusses
\TeX\ and \LaTeX.
Chapter III describes the additional macros and functions
provided by the custom thesis class file.  Finally, Chapter IV hopes to tie things up.
 
It is 
impossible to predict all the formatting problems one will encounter
and there will be problems that are best handled
by a specialist.  
The Graduate School may be able to help you find help.
Some departments may also be able to provide \LaTeX\ assistance.
 
 
\section{Conventions and Notations}
 
In this thesis the typist
refers to the user of \LaTeX---the one who
makes formatting decisions and chooses the appropriate
formatting commands.
He or she will most often be the degree candidate.
 
This document deals with \LaTeX\ typesetting commands and their
functions.  Wherever possible the conventions used to display
text entered by the typist and the resulting formatted output
are the same as those used by the \TeX books.
Therefore, {\tt typewriter type} is used to indicate text
as typed by the computer
or entered by the typist.
It is quite the opposite of {\it italics,} which indicates
a category rather than exact text.  For example,
{\tt alpha} and {\tt beta} might each be an example of a {\it label}.
 
 
\section{Nota bene}
 
This sample thesis was produced by the \LaTeX\ document class it describes
and its format is consonant with the Graduate School's electronic dissertation guidelines,
as of November, 2014, at least.
However, use of this package does not guarantee acceptability
of a particular thesis.
 
 
% ========== Chapter 2
 
\chapter{A Brief \\ Description of \protect\TeX}
 
The \TeX\ formatting program is the creation of
Donald Knuth of Stanford University.
It has been implemented on nearly every general purpose computer and
produces exactly the same copy on all machines.
 
\section{What is it; why is it spelled that way; 
and what do
really long section titles look like in the text and in the
Table of Contents?}
 
\TeX\ is a formatter.  A document's format is controlled
by commands embedded in the text.  
\LaTeX\ is a special version of \TeX---preloaded
with a voluminous set of macros that simplify most
formatting tasks.
 
\TeX\ uses {\it control sequences} to control
the formatting of a document.  These control sequences are usually
words or groups of letters prefaced with the backslash character
({\tt\char'134}).
For example,
Figure \ref{start-2} shows the text that printed the beginning
of this chapter.  Note the control sequence \verb"\chapter" that
instructed \TeX\ to start a new chapter, print the title, and
make an entry in the table of contents.  It is an example
of a macro defined by the \LaTeX\ macro package.
The control sequence \verb"\TeX", which prints the word \TeX,
is a standard macro from the {\it\TeX book}.
The short control sequence \verb"\\" in the title instructed \TeX\ to
break the title line at that point.
This capability is an example of an extension to \LaTeX\
provided by the uwthesis document class.
 
\begin{figure}
\begin{demo}
\uwsinglespace
\\chapter\{A Brief\\\\Description of \\TeX\}

The \\TeX\\ formatting program is the creation of
Donald Knuth of Stanford University.
\end{demo}
\label{start-2}
\caption{The beginning of the Chapter II text}
\end{figure}
 
Most of the time \TeX\ is simply building paragraphs from
text in your source files.  No control sequences are involved.
New paragraphs are indicated by a blank line in the
input file.
Hyphenation is performed automatically.
 
\section{\TeX books}
 
The primary reference for \LaTeX\ is Lamport's second edition
of the \textit{\LaTeX\ User's Guide}\cite{Lbook}.
It is easily read and should be sufficient for thesis formatting.
See also the \textsl{\LaTeX\ Companion}\cite{companion} for descriptions
of many add-on macro packages.

Although unnecessary for thesis writers, the \textsl{\TeX book}
is the primary reference for \TeX sperts worldwide.
 
\section{Mathematics}
 
The thesis class does not expand on \TeX's
or \LaTeX's
comprehensive treatment of mathematical equation printing.%
\label{c2note}\footnote{%
% a long footnote indeed.
 Although many \TeX-formatted documents contain no
 mathematics except the page numbers, it seems appropriate
 that this paper, which is in some sense about \TeX,
 ought to demonstrate an equation or two.  Here then, is a statement 
 of the {\it Nonsense Theorem}.
 
 \smallskip
 \def\RR{{\cal R\kern-.15em R}}
 {\narrower\hangindent\parindent Assume a universe $E$ and a symmetric function
  $\$$ defined on $E$, such that for each $\$^{yy}$ there exists a
  $\$^{\overline{yy}}$, where $\$^{yy} = \$^{\overline{yy}}$.
  For each element $i$ of $E$ define
  ${\cal S}(i)=\sum_i \$^{yy}+\$^{\overline{yy}}+0$.
  Then if $\RR$ is that subset of $E$ where $1+1=3$,
  for each $i$
  $$\lim_{\$\to\infty}\int {\cal S}di =
      \cases{0,&if $i\not\in\RR$;\cr
             \infty,&if $i\in\RR$.\cr}$$
  \par}} % end of the footnote
%
The {\it\TeX book}\cite{book}, {\it \LaTeX\ User's Guide}\cite{Lbook},
and {\it The \LaTeX\ Companion}\cite{companion}
thoroughly cover this topic.
 
 
\section{Languages other than English}
 
Most \LaTeX\ implementations at the University are tailored
for the English language.  However, \LaTeX\ will format many
other languages.  Unfortunately, this author has never been successful in 
learning more than a smattering of anything other than English.
Consult your department or the Tex Users Group.
\smallskip
\begin{center}
{\tt http://tug.org/},
\end{center}
\smallskip
for assistance with non-English formatting.

Unusual characters can be defined via the
font maker \hbox{\mffont METAFONT} (documented by Knuth\cite{Metafont}).
The definitions are not trivial.
Students who attempt to print a thesis with
custom fonts may soon proclaim,
 
% Note.  This is not the ideal way to print Greek
\medskip
\begin{center}
``$\mathaccent"7027\alpha\pi o\kern1pt\theta\alpha\nu\epsilon\hat\iota\nu$
\ $\theta\acute\epsilon\lambda\omega$.''
 
\end{center}
 
% ========== Chapter 3
 
\chapter{The Thesis Unformatted}
 
This chapter describes the uwthesis class (\texttt{uwthesis.cls},
version dated 2014/11/13)
in detail 
and shows how it was used to format the thesis.
A working knowledge of Lamport's \LaTeX\ manual\cite{Lbook} is assumed.
 
\section{The Control File}
 
The source to this sample thesis is a single file
only because ease of distribution was a concern.
You should not do this.  Your task will be much easier if you
break your thesis into several files:  a file for the preliminary pages,
a file for each chapter,  one for the glossary, and one for each
appendix.  Then use a control file to tie them all together.
This way you can edit and format parts of your thesis much more
efficiently.
 
Figure~\ref{control-file} shows a control file that
might have produced this thesis.
It sets the document style, with options and parameters,
and formats the various parts of the thesis---%
but contains no text of its own.
 
 
%  control file caption and figure
%
%
\begin{figure}[p]
 \begin{fullpage}
  \uwsinglespace
  \begin{verbatim}
    % LaTeX thesis control file
 
    \documentclass [11pt, proquest]{uwthesis}[2014/11/13]
 
    \begin{document}
 
    % preliminary pages
    %
    \prelimpages
    \include{prelim}
 
    % text pages
    %
    \textpages
    \include{chap1}
    \include{chap2}
    \include{chap3}
    \include{chap4}
 
    % bibliography
    %
    \bibliographystyle{plain}
    \bibliography{thesis}
 
    % appendices
    %
    \appendix
    \include{appxa}
    \include{appxb}
 
    \include{vita} 
    \end{document}
  \end{verbatim}
  \caption[A thesis control file]%
   {\narrower A thesis control file ({\tt thesis.tex}).
   This file is the input to \LaTeX\ that will produce a
   thesis.  It contains no text, only commands which
   direct the formatting of the thesis.
   }
  \label{control-file}
 \end{fullpage}
\end{figure}
 
The first section, from the \verb"\documentclass" to
the \verb"\begin{document}", defines the document class and options.
This sample thesis specifies the \texttt{proquest} style, which is now
required by the Graduate School and is the default.  
Two other, now dated, other styles are available:  \verb"twoside", which is similar but 
produces a wider binding margin and is more suitable for paper printing; and
\verb"oneside", which is really old fashoned.
This sample also specified a font size
of 11 points. 
Possible font size options are: \verb"10pt", \verb"11pt", and \verb"12pt".
Default is 12 points, which is the preference
of the Graduate School. If you choose a smaller size be sure to
check with the Graduate School for acceptability.  The smaller fonts
can produce very small sub and superscripts.

Include most additional formatting packages with \verb"\usepackage",
as describe by Lamport\cite{Lbook}.  The one exception to this
rule is the \verb"natbib" package.  Include it with the \verb"natbib"
document option.
 
Use the \verb"\includeonly" command to format only a part of your
thesis.  See Lamport\cite[sec. 4.4]{Lbook} for usage and limitations.

 
\section{The Text Pages}
 
A chapter is a major division of the thesis.  Each chapter begins
on a new page and has a Table of Contents entry.
 
\subsection{Chapters, Sections, Subsections, and Appendices}
 
 
Within the chapter title use a \verb"\\" control sequence to separate lines
in the printed title (recall Figure \ref{start-2}.).
The \verb"\\" does not affect the Table of Contents entry.
 
Format appendices just like chapters.
The control sequence \verb"\appendix" instructs \LaTeX\ to
begin using the term `Appendix' rather than `Chapter'.
 
 
Specify sections and subsections of a chapter 
with  \verb"\section" and \verb"\subsection", respectively.
In this thesis chapter and section
titles are written to the table of contents.
Consult Lamport\cite[pg. 176]{Lbook} to see which
subdivisions of the thesis can be written to the table of contents.
The \verb"\\" control sequence is not permitted in section and
subsection titles.
 
 
\subsection{Footnotes}
 
\label{footnotes}
 Footnotes format as described in the \LaTeX\ book.  You can also
 ask for end-of-chapter or end-of-thesis notes.
 The thesis class will automatically set these up if
 you ask for the document class option \texttt{chapternotes}
 or \texttt{endnotes}.  
 
If selected, chapternotes will print automatically.  If you choose
endnotes however you must explicitly indicate when to print the notes 
with the command \verb"\printendnotes".  See the style guide for
suitable endnote placement.  

\subsection{Figures and Tables}
Standard \LaTeX\ figures and tables, see Lamport\cite[sec.~C.9]{Lbook},
normally provide the most convenient means to position the figure.
Full page floats and facing captions are exceptions to this rule.

If you want a figure or table to occupy a full page enclose the
contents in a \texttt{fullpage} environment.  
See figure~\ref{facing-caption}.

\subsubsection{Facing pages}
Facing page captions are an artifact of traditional, dead-tree printing,
where a left-side (even) page faces a right-side (odd) page.

In the \texttt{twoside} style, a facing caption
is full page caption for a full page figure or table
and should face the illustration to which it refers.
You must explicitly format both pages. 
The caption part appears on an even page
(left side) and the figure or table
comes on the following odd page (right side).
Enclose the float contents for the caption 
in a \texttt{leftfullpage} environment,
and enclose the float contents for the figure or table 
in a \texttt{fullpage} environment.
The first page (left side) contains the caption. The second page
(right side) could be left blank.  A picture or graph might be pasted onto
this space. See figure~\ref{facing-caption}.


\begin{figure}[t]
\uwsinglespace
\begin{verbatim}
     \begin{figure}[p]% the left side caption
       \begin{leftfullpage}
         \caption{ . . . }
       \end{leftfullpage}
     \end{figure}
     \begin{figure}[p]% the right side space
       \begin{fullpage}
          . . .
          ( note.. no caption here )
       \end{fullpage}
     \end{figure}
\end{verbatim}
\caption[Generating a facing caption page]{This text would create a
  double page figure in the two-side styles. }
\label{facing-caption}
\end{figure}
 
You can use these commands with the \texttt{proquest} style, but they have little
effect on online viewing.
 
 
\subsection{Horizontal Figures and Tables}
Figures and tables may be formatted horizontally
(a.k.a.\ landscape) as long as their captions appear
horizontal also.  \LaTeX\ will format landscape material for you.

Include the \texttt{rotating} package 
\begin{demo}
\\usepackage[figuresright]\{rotating\}
\end{demo}
and read the documentation that comes with the package. 

Figure~\ref{sideways} is an example of how a landscape
table might be formatted. 

\begin{figure}[t]
\uwsinglespace
\begin{verbatim}
     \begin{sidewaystable}
         ...
         \caption{ . . . }
     \end{sidewaystable}
\end{verbatim}
\caption[Generating a landscape table]{This text would create a
  landscape table with caption.}
\label{sideways}
\end{figure}
 


\subsection{Figure and Table Captions}
Most captions are formatted with the \verb"\caption" macro as described 
by Lamport\cite[sec. C.9]{Lbook}. 
The uwthesis class extends this macro to allow
continued figures and tables, and to provide multiple figures and
tables with the same number, e.g., 3.1a, 3.1b, etc.
 
To format the caption for the first part of
a figure or table that cannot fit
onto a single page use the standard form:
\begin{demo}
\\caption[\textit{toc}]\{\textit{text}\}
\end{demo}
To format the caption for the subsequent parts of 
the figure or table 
use this caption:
\begin{demo}
\\caption(-)\{(continued)\}
\end{demo}
It will keep the same number and the text of the caption will be 
{\em(continued)}.

To format the caption for the first part of
a multi-part figure or table
use the format:
\begin{demo}
\\caption(a)[\textit{toc}]\{\textit{text}\}
\end{demo}
The figure or table will be lettered (with `a') as well as numbered.
To format the caption for the subsequent parts of 
the multi-part figure or table
use the format:
\begin{demo}
\\caption(\textit{x})\{\textit{text}\}
\end{demo}
where {\em x} is {\tt b}, {\tt c}, \ldots.
The parts will be lettered (with `b', `c', \ldots).

If you want a normal caption, but don't want a ToC entry:
\begin{demo}
\\caption()\{\textit{text}\}
\end{demo}
Note that the caption number will increment.  You would normally use this 
only to leave an entire chapter's captions off the ToC.


\subsection{Line spacing}

Normally line spacing will come out like it should. However, the 
ProQuest style allows single spacing in certain situations:
figure content, some lists, and etc.
Use \verb"\uwsinglespace" to switch to single spacing within
a \verb"\begin{}" and \verb"\end{}" block.
The code examples in this document does this. 

\section{The Preliminary Pages}
 
These are easy to format only because they are relatively invariant
among theses.  Therefore the difficulties have already been encountered
and overcome by \LaTeX\ and the thesis document classes.

Start with the definitions that describe your thesis.
This sample thesis was printed with the parameters:

\begin{demo}
\\Title\{The Suitability of the \\LaTeX\\ Text Formatter\\\\
   for Thesis Preparation by Technical and\\\\
   Non-technical Degree Candidates\}
\\Author\{Jim Fox\}
\\Program\{IT Infrastructure\}
\\Year\{2012\}

\\Chair\{Name of Chairperson\}\{title\}\{Chair's department\}
\\Signature\{First committee member\}
\\Signature\{Next committee member\}
\\Signature\{etc\}

\end{demo}
 
Use two or more \verb"\Chair" lines if you have co-chairs.
 
\subsection{Copyright page}
Print the copyright page with \verb"\copyrightpage".

\subsection{Title page}
Print the title page with \verb"\titlepage".
The title page of this thesis was printed with%
 
\begin{demo}
\\titlepage
\end{demo}
 
You may change default text on the title page with these
macros.  You will have to redefine \verb"\Degreetext", for instance,
if you're writing a Master's thesis instead of a dissertation.\footnote{If you use these they can
be included with the other information before \\copyrightpage".}

\begin{list}{}{\itemindent\parindent\itemsep0pt
   \def\makelabel#1{\texttt{\char`\\#1}\hfill}\uwsinglespace}
\item[Degree\char`\{{\it degree name}\char`\}]
   defaults to ``Doctor of Philosophy''
\item[School\char`\{{\it school name}\char`\}] defaults to
``University of Washington''
\item[Degreetext\char`\{{\it degree text}\char`\}] defaults to
``A dissertation submitted \ldots''
\item[textofCommittee\char`\{{\it committee label}\char`\}] defaults to
``Reading Committee:''
\item[textofChair\char`\{{\it chair label}\char`\}] defaults to
``Chair of the Supervisory Committee:''
\end{list}

These definitions must appear \underline{before} the \verb"\titlepage" command.

 
\subsection{Abstract}
Print the
abstract with \verb"\abstract".
It has one argument, which is the text of the abstract.
All the names have already been defined.
The abstract of this thesis was printed with
 
\begin{demo}
\\abstract\{This sample . . . `real' dissertation.\}
\end{demo}
 
 
\subsection{Tables of contents}
Use the standard \LaTeX\ commands to format these items.
 
 
\subsection{Acknowledgments}
Use the \verb"\acknowledgments" macro to format the acknowledgments page.
It has one argument, which is the text of the acknowledgment.
The acknowledgments of this thesis was printed with
 
\begin{demo}
\\acknowledgments\{The author wishes . . . \{\\it il miglior fabbro\}.\\par\}\}
\end{demo}
 
 
\subsection{Dedication}
Use the \verb"\dedication" macro to format the dedication page.
It has one argument, which is the text of the dedication.
 
\subsection{Vita}
Use the \verb"\vita" macro to format the curriculum vitae.
It has one argument, which chronicles your life's accomplishments.

Note that the Vita is not really a preliminary page.
It appears at the end of your thesis, just after the appendices.
 
 
%%  
%% \section{Customization of the Macros}
%%  
%% Simple customization, including 
%% alteration of default parameters,  changes to dimensions,
%% paragraph indentation, and margins, are not too difficult.
%% You have the choice of modifying the class file ({\tt uwthesis.cls})
%% or loading
%% one or more personal style files to customize your thesis.
%% The latter is usually most convenient, since you do not need
%% to edit the large and complicated class file.
%% 
 


% ========== Chapter 4
 
\chapter{Running \LaTeX\\
  ({\it and printing if you must})}
 
 
From a given source \TeX\ will produce exactly the same document
on all computers and, if needed, on all printers.  {\it Exactly the same}
means that the various spacings, line and page breaks, and
even hyphenations will occur at the same places.

How you edit your text files and run \LaTeX\ varies
from system to system and depends on your personal preference.

\section{Running}

The author is woefully out of his depth where 
\TeX\ on Windows is concerned.  Google would be his resource.
On a linux system he types

\begin{demo}
\$\ pdflatex uwthesis
\end{demo}

and it generally works.

 
\section{Printing}
 
All implementations of \TeX\ provide the option of {\bf pdf} output,
which is all the Graduate School requires.  Even if you intend to
print a copy of your thesis create a 
{\tt pdf}.  It will print most anywhere.

\printendnotes

%
% ==========   Bibliography
%
\nocite{*}   % include everything in the uwthesis.bib file
\bibliographystyle{plain}
\bibliography{uwthesis}
%
% ==========   Appendices
%
\appendix
\raggedbottom\sloppy
 
% ========== Appendix A
 
\chapter{Where to find the files}
 
The uwthesis class file, {\tt uwthesis.cls}, contains the parameter settings,
macro definitions, and other \TeX nical commands which
allow \LaTeX\ to format a thesis.  
The source to
the document you are reading, {\tt uwthesis.tex},
contains many formatting examples
which you may find useful.
The bibliography database, {\tt uwthesis.bib}, contains instructions
to BibTeX to create and format the bibliography.
You can find the latest of these files on:

\begin{itemize}
\item My page.
\begin{description}
\item[] \verb%https://staff.washington.edu/fox/tex/thesis.shtml%
\end{description}

\item CTAN
\begin{description}
\item[]  \verb%http://tug.ctan.org/tex-archive/macros/latex/contrib/uwthesis/%
\item[]  (not always as up-to-date as my site)
\end{description}

\end{itemize}

\vita{Jim Fox is a Software Engineer with IT Infrastructure Division at the University of Washington.
His duties do not include maintaining this package.  That is rather
an avocation which he enjoys as time and circumstance allow.

He welcomes your comments to {\tt fox@uw.edu}.
}


\end{document}
