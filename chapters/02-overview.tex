\chapter{Overview}
\label{sec:overview}

Term rewriting systems~\cite{gorn1967} are sets of \textit{rewrite rules} used to transform expressions into a new form.  Such systems are widely
used in theorem proving~\cite{baader1999term} and abstract interpretation~\cite{cousot1977abstract, cousot1979systematic}.

Terms are defined inductively over a set of variables $V$ and a set of function symbols $\Sigma$. Every variable $v \in V$ is a term, and for any function symbol $f \in \Sigma$ with arity $n$ and any terms $t_1, ..., t_n$, the application of the symbol to the terms $f(t_1, ..., t_n)$ is also a term. (Constants are considered zero-arity functions.) We refer to the set of terms constructed from the variables $V$ and the function symbols $\Sigma$ as $T(\Sigma, V)$.

A \emph{rewrite rule} is a directed binary relation $l \rewrites r$ such that $l$ is not a variable, and all variables present in $r$ are also present in $l$ (i.e., $\mathcal{V}ar(l) \supseteq \mathcal{V}ar(r)$). A set of rewrite rules is called a \emph{term rewriting system}.

Consider a set of terms $T(\Sigma, V)$ such that $\Sigma = \{\clubsuit, \diamondsuit\}$ and $V$ is an infinite set of variables. Let the term rewriting system $R$ consist of a single rule:

\[ R = \{ x_1 \clubsuit x_2 \rewrites x_1 \diamondsuit x_2 \} \]
We use $R$ to rewrite the term

\[ 
(y_1 \diamondsuit y_1) \clubsuit (y_2 \clubsuit y_3)
\]
The first step is matching; we find a substitution that will unify the left-hand side (LHS) of the rule with the term we are rewriting. Here, one possible substitution is:

\[
\{ x_1 \mapsto (y_1 \diamondsuit y_1), x_2 \mapsto (y_2 \clubsuit y_3) \}
\]
We then apply this substitution to the right-hand side (RHS) of the rule to obtain the rewritten version of the original term:

\[ 
(y_1 \diamondsuit y_1) \diamondsuit (y_2 \clubsuit y_3)
\]