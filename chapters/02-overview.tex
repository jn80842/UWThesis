\chapter{Overview}
\label{chapter:overview}

\jln{this refers only to the simplifier, expand to include variable solver.}
The Halide compiler contains a term rewriting system composed of over a thousand
rules, operating over the space of Halide expressions\footnote{See Supplemental Material for
the full Halide expression grammar}.  \modified{The language of Halide expressions
operates over vectors and scalars of integers, booleans, and real values.  However, in
this work we concentrate on the TRS as it applies to integer and boolean values, for
both vectors and scalars, because the most important uses of the TRS within the compiler
apply to these types.  In this section, we give some background on term rewriting systems.
We then describe how the Halide compiler uses the TRS
and the design decisions that motivate the custom rewriting engine, as well as the scope
of our work.}

\section{Term Rewriting Systems}

Term rewriting systems~\cite{gorn1967} are sets of \textit{rewrite rules} used to transform expressions into a new form.  Such systems are widely
used in theorem proving~\cite{baader1999term} and abstract interpretation~\cite{cousot1977abstract, cousot1979systematic}.

Terms are defined inductively over a set of variables $V$ and a set of function symbols $\Sigma$. Every variable $v \in V$ is a term, and for any function symbol $f \in \Sigma$ with arity $n$ and any terms $t_1, ..., t_n$, the application of the symbol to the terms $f(t_1, ..., t_n)$ is also a term. (Constants are considered zero-arity functions.) We refer to the set of terms constructed from the variables $V$ and the function symbols $\Sigma$ as $T(\Sigma, V)$.

A \emph{rewrite rule} is a directed binary relation $l \rewrites r$ such that $l$ is not a variable, and all variables present in $r$ are also present in $l$ (i.e., $\mathcal{V}ar(l) \supseteq \mathcal{V}ar(r)$). A set of rewrite rules is called a \emph{term rewriting system}.

Consider a set of terms $T(\Sigma, V)$ such that $\Sigma = \{\clubsuit, \diamondsuit\}$ and $V$ is an infinite set of variables. Let the term rewriting system $R$ consist of a single rule:

\[ R = \{ x_1 \clubsuit x_2 \rewrites x_1 \diamondsuit x_2 \} \]
We use $R$ to rewrite the term

\[ 
(y_1 \diamondsuit y_1) \clubsuit (y_2 \clubsuit y_3)
\]
The first step is matching; we find a substitution that will unify the left-hand side (LHS) of the rule with the term we are rewriting. Here, one possible substitution is:

\[
\{ x_1 \mapsto (y_1 \diamondsuit y_1), x_2 \mapsto (y_2 \clubsuit y_3) \}
\]
We then apply this substitution to the right-hand side (RHS) of the rule to obtain the rewritten version of the original term:

\[ 
(y_1 \diamondsuit y_1) \diamondsuit (y_2 \clubsuit y_3)
\]


\section{Term rewriting systems in the Halide compiler}

\jln{again, all specific to simplifier; add variable solver content}
\subsection{\modified{Uses of the TRS in Halide}}
\label{sec:uses-of-trs}
\modified{While the Halide compiler makes use of the TRS in numerous ways, the most important
  applications of the TRS are its uses as a fast simplifier and as a proof engine.  
  In many parts of the compiler, the TRS is used to rewrite expressions into simpler forms,
  which are easier for the compiler to reason about, and result in less code being generated for
  LLVM to consume at the backend.  Most importantly, the compiler uses the TRS to simplify expressions
  into constants or expressions that are monotonic with respect to loop bounds; these simplifications are core to Halide's
  ability to generate drastically different loop nests for different schedules.}

\modified{For example, consider the simple two-stage imaging pipeline $g(x) = f(x - 1) + f(x) + f(x + 1)$.
  Halide enables programmers to fuse the computation of $f$ into $g$ at an arbitrary granularity
  using the \texttt{compute\_at} scheduling directive.  This requires Halide to automatically reason
  about which region of $f$ is required for a specific sub-region (or tile) of $g$, using interval
  arithmetic over symbolic values for the size of a tile of $g$.  For a tile size of 8, a tile of $g$
  is the region \texttt{[g.tile\_min, g.tile\_min+7]};  the region of $f$ required is
  \texttt{[g.tile\_min-1, g.tile\_min+8]}; and the number of values of $f$ to compute is then
  \texttt{g.tile\_min + 8 + 1 - (g.tile\_min - 1)}.  If the TRS can determine this is a static value
  of 10, the Halide compiler can then safely perform transformations requested by the user.  In this
  case, the compiler can use stack memory instead of inserting a dynamic allocation; or the loop can be
  completely unrolled; the loop can be vectorized; or $f$ can be mapped to GPU threads (since a single
  threadblock must have a compile-time-known size).  More generally, this kind of region analysis
  operates most effectively when the expressions are monotonic in the loop bounds; otherwise, interval
  arithmetic can result in vast overestimates of required regions.  These simplifications are essential
  for the compiler to work, and are usually not as simple as this example.}

\modified{The rules for simplifying to perform cancellations and
  ensure monotonicity are incredibly important for compiler
  performance. When we disabled all but the constant-folding rules to
  measure the importance of the simplifier, it was the absence of
  these specific rules that caused the (26.4$\times$) slow-down
  mentioned in Section~\ref{sec:introduction}. Without these rules,
  Halide is useless for high-performance image
  processing.}

%% Fodder from synthesis section:
%% For example, Halide relies on symbolic interval arithmetic to determine how much memory
%% to allocate and how many values to compute for each stage.  Symbolic interval arithmetic
%% is exact when an expression monotonically increases or decreases over a loop.}
%% For example, if $x \in [0, 100]$, then symbolic interval arithmetic states that
%% $\hmax(x, x/2 + 20) \in [20, 100]$, which is the tightest correct
%% bound; this bound is obtained by substituting in the lower and upper bounds of $x$
%% into the expression. However, in the presence of anti-correlated subexpressions
%% interval arithmetic becomes inexact, and is prone to overestimating
%% bounds. The expression $\hmin(x, 100 - x)$ when $x \in [0, 100]$ is bounded above by 50, but
%% symbolic interval arithmetic makes the weaker claim that it is bounded
%% above by 100, by setting the first instance of $x$ to 100 and the
%% second instance of $x$ to zero.

\modified{The use as a proof engine occurs when the compiler must prove properties about the code in order to guarantee the
  correctness of specific transformations or the relationships between bounds of
  different loops or producer-consumer relationships.  In such cases, the compiler constructs
  an expression that must be true or false in order to guarantee correctness, then applies
  the TRS to see if the expression simplifies to a single boolean value.}

\modified{For example, Halide uses Euclidean division, which rounds according to the sign of the
  denominator.  Lowering this to code requires emitting several instructions, which can be
  slower than native division.  When the compiler can statically prove the signs of the numerator
  and denominator, in some cases the code can be replaced by native division or even a different
  instruction altogether.  For example, for an expression \texttt{x / max(y, 1)} the compiler
  will try to prove $0 < \hmax(y, 1)$.  The TRS first invokes a rule to transform this to
  $0 < y\; ||\; 0 < 1$, which then is transformed to true (since the second clause is always true).
  Thus, the compiler is able to replace Euclidean division with machine division.}

\modified{TRS failures have adverse results on the compiler, making it unpredictable and
  difficult for programmers to use.  When the TRS fails to properly simplify an expression or
  prove a property, the consequences include: }
\begin{itemize}
\item Insufficiently tight bounds on loops and allocations, which may result in
  runtime failures (e.g. due to memory overallocation) or performance issues;

\item Failure of the compiler to apply optimizations, also resulting in slow performance;

\item Dynamic checks in the generated code for properties that could have been proven
  at compile time, leading to slower code;

\item Compilation failures, when the compiler is unable to correctly produce code
  even though the properties required hold, or when the proof engine itself crashes
  or loops infinitely.
\end{itemize}

Thus, correctness and generality of the TRS are essential to make the compiler
robust and able to generate fast code.

\begin{figure*}
\begin{tabular}{cccc}

%\Tree [.+ [.- [.min a b ] [.max c c ] ] [.max c c ]]
\begin{tikzpicture}[level distance=12mm]
\tikzstyle{level 1}=[sibling distance=15mm]
\tikzstyle{level 2}=[sibling distance=10mm]
\tikzstyle{level 3}=[level distance=10mm,sibling distance=5mm]


%\Tree [.+ [.- [.min a b ] c ] [.max c c ]]
\node (+) {+}
  child { node (-) {-}
    child { node (min) {\hmin}
      child {node (a) {a}}
      child {node (b) {b}}
    }
    child { node (max2) {\hmax}
      child {node (c4) {c}}
      child {node (c5) {c}}
    }
  }
  child { node (max) {\hmax}
    child { node (c2) {c}}
    child { node (c3) {c}}
  };
\begin{pgfonlayer}{background}
\fill[red,opacity=0.3] \convexpath{c4,max2,c5}{10pt};
\end{pgfonlayer}
\end{tikzpicture}

&
\begin{tikzpicture}[level distance=12mm]
\tikzstyle{level 1}=[sibling distance=15mm]
\tikzstyle{level 2}=[sibling distance=7mm]
\tikzstyle{level 3}=[level distance=10mm,sibling distance=5mm]
%\Tree [.+ [.- [.min a b ] c ] [.max c c ]]
\node (+) {+}
  child {  node (-) {-}
    child { node (min) {\hmin}
      child {node (a) {a}}
      child {node (b) {b}}
    }
    child { node (c) {c}}
  }
  child { node (max) {\hmax}
    child { node (c2) {c}}
    child { node (c3) {c}}
  };
\begin{pgfonlayer}{background}

\fill[blue,opacity=0.3] \convexpath{c2,max,c3}{10pt};
%\fill[red,opacity=0.3] \convexpath{c}{10pt};
\draw[fill=red,opacity=0.3,draw=none](c) circle (10pt);

\end{pgfonlayer}
\end{tikzpicture}
&
\begin{tikzpicture}[level distance=12mm]
\tikzstyle{level 1}=[sibling distance=15mm]
\tikzstyle{level 2}=[sibling distance=10mm]
\tikzstyle{level 3}=[level distance=10mm,sibling distance=5mm]
%\Tree [.+ [.- [.min a b ] c ] c ]
\node (+) {+}
  child { node (-) {-}
    child { node (min) {\hmin}
      child {node (a) {a}}
      child {node (b) {b}}
    }
    child { node (c) {c} }
  }
  child { node (c2) {c}};



  \begin{pgfonlayer}{background}
\fill[green,opacity=0.3] \convexpath{b,a,min,-,+,c2,c}{10pt};
\draw[fill=blue,opacity=0.3,draw=none](c2) circle (10pt);
%\draw[red,fill=blue,opacity=0.3](c2.north) to[closed,curve through={($(c2.north east)!1.0!(c2.south east)$) .. ($(c2.south west)!1.0!(c2.north west)$)}] (c2.north);
\end{pgfonlayer}
\end{tikzpicture}
&
\vspace{0pt}
\begin{tikzpicture}[level distance=12mm]
\tikzstyle{level 1}=[sibling distance=15mm]
\tikzstyle{level 2}=[sibling distance=10mm]
\tikzstyle{level 3}=[level distance=10mm,sibling distance=5mm]
\node (min) {\hmin}
  child { node (a) {a}
     % [red,opacity=0.0]
    child {     [red,opacity=0.0] node (fake1) {f}
      child {    [red,opacity=0.0] node (fake2) {f}}
      child {    [red,opacity=0.0] node (fake3) {f}}
    }
    child {     [red,opacity=0.0] node (fake4) {f}
    [red,opacity=0.0]
    child { [red,opacity=0.0] node (fake) {f}}
    child { [red,opacity=0.0] node (fake2) {f}}
    }
  }
  child { node (b) {b}
     % [red,opacity=0.0]
    child {     [red,opacity=0.0] node (fake5) {f}}
    child {     [red,opacity=0.0] node (fake6) {f}}
  };
  \begin{pgfonlayer}{background}
\fill[green,opacity=0.3] \convexpath{a,min,b}{10pt};
\end{pgfonlayer}

\end{tikzpicture} \\
(i) & (ii) & (iii) & (iv)
\end{tabular}
\caption{We demonstrate the Halide rewriting algorithm using a TRS $R = \{\hmax(x, x) \rewrites x, (x - y) + y \rewrites x\}$ and an expression $\hmin(a,b) - \hmax(c,c) + \hmax(c,c)$. The algorithm attempts to simplify all subtrees bottom up; here, no rule applies to $\hmin(a,b)$ so it is not changed. Next (i), rule 1 rewrites $\hmax(c,c)$ to $c$. No rule applies to $\hmin(a,b) - c$, so we move to the rightmost subtree and rewrite again (ii) to obtain $c$ from $\hmax(c,c)$. Finally, we consider the entire tree $\hmin(a,b) - c) + c$ (iii) and apply rule 2 to produce $\hmin(a,b)$. No rules match this expression, so we are left with $\hmin(a,b)$ (iv).}
\label{fig:algoexample}
\end{figure*}

\subsection{Why a Custom Algorithm?}
\label{sec:whycustom}

\modified{In a term rewriting system, a single rule may be able to match an input expression in 
multiple ways, and there may be multiple rules in the ruleset which could be used 
to rewrite the expression. A term rewriting algorithm might choose one of many alternatives 
and later backtrack if it turns out not to be fruitful; it might make use of 
heuristics to choose a next step; it might exercise all the alternatives and keep 
the results in equivalence classes, as in an egraph. The Halide term rewriting algorithm
keeps only one expression in state and applies rules greedily, in a fixed priority.
This is very fast and requires very little memory; the tradeoff is that the algorithm 
may pick the ``wrong'' rule and have no way of undoing that decision. 
Since the rewriter is invoked thousands of times with each call of the compiler, 
it chooses to sacrifice some solving power in exchange for performance.}

\subsubsection{\modified{Halide's Custom TRS Algorithm}}
\label{sec:customalgo}
The Halide term rewriting algorithm simplifies an input expression in a
depth-first, bottom-up traversal of the expression DAG. At each node, it 
uses the root node to pick a list of rules, then
attempts to match the subtree expression with the rule LHSs in a fixed priority. Matching
is performed purely syntactically, using C++ template metaprogramming.
 \modified{Halide rewrite rules contain special metavariables,
called \emph{symbolic constants}, that can match only with constant values; all other
variables can match any subterm as usual.}
When a match is found, the algorithm rewrites the
subtree expression using the RHS of that rule, and then attempts to simplify the
subtree expression again. If no rule matches the subtree, the traversal
continues; when the entire expression cannot be simplified further, the
rewritten expression is returned. See Figure~\ref{fig:algoexample} for a worked
example.

The rewrite rules optionally contain a compile-time predicate guard. 
These guards contain only symbolic constants\footnote{Existing 
rules sometimes have predicates that check if
  non-constant variables can be shown to have certain properties at compile
  time, but these are expensive and used sparingly.}; when the LHS of a rule
matches an expression, its guard is evaluated and only if it
is true will the rewrite be applied.

\modified{Halide rewrite rules are applied in a fixed priority, organized so that the TRS 
first attempts very basic rules such as constant folding, then tries more specific 
rules before more general ones. (We do not evaluate the current rule priority in this work.)}

\modified{Associativity and commutativity laws are particularly troublesome for term rewriting systems. 
For one expression $e$, the number of semantically equivalent expressions grows 
exponentially in terms of the number of AC operations $e$ contains. Some term rewriting 
systems perform a full \emph{AC matching} step during rewriting. Halide's TRS does not
perform this matching, but instead
includes multiple AC variations of rules.
However, a small number of Halide's rewrite rules have the effect of canonicalizing some commutative
expressions. (For example, if a commutative expression has a multiplication as its 
first operand and a subtraction as its second operand, a rule will switch their positions.)
These rules are all early in the application priority, so later rules can rely on
expressions having a quasi-canonical form.}

\subsubsection{Why Not Z3?}
\begin{table}
\caption{We compare the performance of Z3 and the Halide TRS in proving a set of 4304 expressions gathered from realistic compiler output. Note that expressions in the ``not proven'' column include expressions that are true but not found to be so by the solvers as well as expressions that are not true. }
\begin{tabular}{l|r|r|r}
Tool & Runtime & Proven expressions & Not proven \\
\hline
Z3 & 7m29s & 1125 & 3179 \\
Halide TRS & 2s & 885 & 3419 
\end{tabular}
\label{tab:simplifiervsz3}
\end{table}


Given that we make use of the Z3 solver~\cite{de2008z3} for both verification and synthesis, it is natural to ask why Halide could not simply call Z3 for simplification. Z3 is the product of extensive development and is a very powerful, general-purpose solver. However, the Halide term rewriting system has a few key properties that Z3 does not: deterministic output, low memory and compute requirements, and domain-specific optimizations.

As discussed above, the Halide compiler must return the same schedule every time the same pipeline is run. Z3 can fix a random seed, but long-running queries may complete on a more powerful server while timing out on a different machine.

\modified{While the Halide algorithm is less powerful than Z3, its deterministic, greedy rule application strategy
gives it a smooth performance curve, whether it succeeds or fails in simplifying an input expression. }
A solver like Z3 tends to give very good performance most of the time but gets bogged down in difficult cases, requiring the use of timeouts. The Halide algorithm ``fails fast'': on an input expression which does not match any rule,  the Halide algorithm will complete in time linear to the size of the expression, taking on the order of one CPU cycle per term in the expression per rule in the TRS. To demonstrate this performance tradeoff, we gathered 4304 expressions from queries the Halide compiler made when compiling realistic pipelines, including provably true expressions and expressions that are not provably true. Z3 could prove more expressions true (within a 60 second timeout), but was starkly less performant. As shown in Table~\ref{tab:simplifiervsz3}, Z3 took over 7 minutes to check the set of expressions while the Halide TRS took just 2 seconds. This set of expressions is much smaller than the number of calls the compiler makes to the rewriter in compiling a single pipeline.

Because the Halide algorithm at every step chooses one rule to apply to the single expression it is working on, it scales well in terms of the number of rules in the TRS. See Section~\ref{ssec:compilationspeed} for an evaluation of the effects of adding newly synthesized rules on the performance of the compiler. 

Finally, although Z3 can simplify expressions, simply reducing the size of an expression is not necessarily the goal for the compiler. For example, gathering like terms in some cases can actually prevent Halide or LLVM optimizations from applying. The Halide rewriter uses a domain-specific strategy to guide expressions into more optimizable forms and can be changed or tuned as needed if further optimizations are discovered. 

\subsection{\modified{Completeness of the Halide TRS}}
\label{sec:completion}

\modified{We know by observation that the current Halide TRS cannot prove certain equalities 
that are in fact true, or reduce certain expressions that can be further simplified. 
Our goal is to learn new rules that would strengthen the TRS and allow it to make
further progress on these ``stuck'' expressions. This goal seems similar to that of 
\emph{completion}, which constructs a decision procedure through syntatic rewrites
for a set of identities. We do not use completion directly, although
our synthesis algorithm could be considered analogous to completion in some ways.}

\modified{In the standard Knuth-Bendix completion algorithm~\cite{knuth1983simple}, we take a
finite set of identities $E$ and a reduction order $>$ on terms as inputs; if successful, the
algorithm will return a finite, convergent set of rules $R$ that is equivalent to 
$E$. The algorithm may also fail, or fail to terminate. At each step the algorithm
maintains a set of identities and a set of rules, both of which can be updated; the 
algorithm may transform an identity into a new rule, find a new identity as a 
consequence of the ruleset, or use the present ruleset to refine either an identity 
or a rule. The algorithm runs until the ruleset has converged; specific implementations
may use some conditions under which to terminate with a failure.}

\modified{No finite set of identities exists for the theory of integers. We could
fix a set of identities to use in a completion procedure, but choosing these axioms
is a non-trivial task. One issue is that the theory contains axioms such as commutativity;
an identity such as $x + y \equiv y + x$ cannot be oriented by any possible reduction 
order, so our completion algorithm cannot make use of this fact. Another is that any
sufficiently powerful set of identities would almost certainly result in a non-terminating
completion procedure.  In addition, even if we use a subset of the Halide TRS for our
identities (thus yielding a confluent Halide TRS), our experience shows that many
failures in the compiler's use of the TRS are due to missing semantic facts that are not
derivable from the current Halide ruleset.}

%%% SAK: I integrated this into a sentence above.  Not sure we need this much text.
%% \modified{One approach might be to transform the existing Halide TRS into a set of undirected
%% identities $E_R$ and use it together with a reduction order as inputs to a completion procedure. 
%% If successful, this would yield a confluent TRS, for which any sequence of rewrites to 
%% an expression would ultimately yield the same result. (It is not immediately clear how 
%% to handle rule predicates in this case, however.) While this is certainly desirable, 
%% our experience is that many proof failures in the current Halide TRS is not due to a 
%% lack of confluence, but because of knowledge of the Halide expression langauge semantics 
%% not encompassed by the identities in $E_R$.}

\modified{%Instead, our synthesis algorithm seeks to learn rules which cannot (necessarily) be 
  %deduced from the existing ruleset.
In the absence of a finite set of identities, we
treat an SMT solver as a decision procedure to determine if a suggested identity 
holds in our theory (of course, the solver itself is also incomplete; we only make use of the 
soundness of the solver and cannot derive any information in the case where the solver cannot
show that an identity holds). If the identity holds and can be oriented using our reduction order, 
it is added as a rule. It is possible that the newly-synthesized rule may be a consequence 
of the existing ruleset and thus could have been found by running completion, 
but we know that many synthesized rules contain information that is previously 
unknown to the TRS.}

\modified{If we consider our solver as standing in for an infinite set of identities that make up
our theory, we clearly could synthesize an infinite number of rules. Here we make use
of the fact that expressions encountered by the compiler have some bias 
and are not sampled randomly from the entire expression space. In a preliminary 
experiment, we tried generating LHSs at random within a certain expression size and 
synthesizing RHSs to serve as new rules. We were able to find an extremely large number of 
``missing'' rules not represented in the current TRS, but the new ruleset had 
no measurable performance impact on benchmarks. Thus, we only synthesize rules if their LHS could be 
applied to at least one expression observed by the compiler under realistic usecases. %Even 
%without a formal definition of the bias on encountered expressions, we are able to 
%target synthesis to produce new rules that have some impact on compiler output.
}

\subsection{Scope: Robust, Fast, Non-Backtracking Ruleset}
\modified{In this work, we operate within the scope of Halide's TRS algorithm
  and work to make the TRS as correct, general, and robust as possible.  Because
  the space of expressions we consider constitutes an undecidable theory, a complete
  TRS is impossible.  Instead, we strive to improve correctness by ensuring the TRS
  will always terminate on any expression and that each individual rewrite
  preserves semantics; and we improve generality by expanding the ruleset to
  contain rewrites that apply to real-world expressions, rather than
  arbitrary new rules that may not apply to any expressions the compiler will encounter.}

\modified{These improvements require overcoming challenging obstacles.  First,
  we must perform a post-hoc verification of a large body of existing rules;
  proving a subset of rules correct or that a subset of rules do not result in infinite
  rewriting loops is insufficient to guarantee robust behavior.  Secondly,
  these rewrites operate in an undecidable theory, making automated verification
  difficult.  Finally, because of this undecidability, we cannot necessarily
  rely on traditional automated techniques to discover new rules.}
%\sak{Should we say something about how this applies to things other than Halide?}

\subsubsection{TrimNoOps}

When compiling a pipeline, the Halide compiler will attempt to truncate loop bounds to the regions where work is actually performed. The TrimNoOps phase examines a for loop and attempts to prove that some regions of the loop are no-ops; if it can show this, it rewrites the loop so those regions are never visited. The variable solver used in performing this reasoning.

For example, consider a for loop that applies the following function over some range of $x$ and $y$:

\begin{verbatim}
f(x, y) = x + y + select((x == 10) && (x < y), 1, 0)
\end{verbatim}

After several compilation phases, the for loop is transformed into:

\begin{verbatim}
f = {
	for(y, y_min, y_extent) {
		for(x, x_min, x_extent) {
			f[(y * stride + x) - (y_min * stride + x_min)] = x + y
		}
	}
	for(y, y_min, y_extent) {
		for(x, x_min, x_extent) {
			f[(y * stride + x) - (y_min * stride + x_min)] = 
			f[(y * stride + x) - (y_min * stride + x_min)] + select((x == 10) && (x < y), 1, 0)
		}
	}
}
\end{verbatim}

%%% the example itself is easy to understand but it doesn't actually make use of the variable solver in solving it (it does call it, just doesn't get back interesting results)


\section{Related work}
\label{sec:related}
\jln{putting this here from generals doc, but doesn't really belong here}
Perhaps the closest related work is the Alive project~\cite{lopes2015alive,menendez2017aliveinfer}.
The fundamental difference between Alive and this
work is that Alive works within the decidable theory of bitvectors, while
(because of Halide semantics) we must use the undecidable theory of integers;
this constraint is the major reason for many of our design choices. In addition:
Alive verifies optimizations (and Alive-Infer synthesizes preconditions), while
we synthesize rewrites and predicates, as well as verify them; Alive must contend
with more types of undefined behavior, which the Halide expression
language need not consider; and Alive uses a simple reduction order in
which all optimizations reduce program size, while our termination proof is more
complex. We originally tried synthesizing rule predicates with the approach used
by Alive-Infer but were not successful: using Z3 to generate positive and
negative examples did not scale for us, requiring seconds to minutes per query
due to the underlying theory of integers.  Moreover, queries with
division/modulo over the integers often did not work at all, simply returning
``unknown.''

Most recently, leveraging a TRS along with synthesized rules has been applied
to optimizing fully-homomorphic encryption (FHE) circuits~\cite{lee2020fhe}.  This
system synthesizes equivalent circuits with lower cost from small example circuits,
then applies the equivalences in a divide-and-conquer manner; the rewrites do
not contain preconditions. In further contrast to our work, the domain of FHE yields a
simple cost function (the depth of nested multiplications in the circuit), and
the underlying theory of boolean circuits is decidable.


An \textit{equivalence graph} or egraph (introduced by \cite{nelson1980techniques}, see \cite{willsey2021egg} for a recent treatment) is a data structure used to compute applications of the rules of a term rewriting system. The algorithm builds up equivalence classes by successively applying all rules to all expressions within those classes, then queries to see if two expressions are equivalent by checking if they are present in the same class. Like our algorithm, it does not backtrack, but the egraph can require significant amounts of memory, which our algorithm avoids.

Herbie~\cite{panchekha2015automatically}, a tool for improving the accuracy of floating point arithmetic, uses an egraph term rewriting system made up of a small library of axioms to find repairs once a fault has been localized. Herbie assures termination by bounding the number of rewrites their system may apply, and achieves good performance by pruning the expression search space and applying rewrites only to particular expression nodes. 

Besides the closely-related projects described above, program synthesis has been applied to term rewriting systems in several domains. Swapper~\cite{singh2016swapper} synthesizes a set of rewrite rules to transform SMT formulas into forms that can be more easily solved by theory solvers, similar to the use of the Halide TRS as a simplifier, using the SKETCH tool. \citet{butler2017synthesizing} learns human-interpretable strategies (essentially rewrite rules) for puzzle games such as Sudoku or Nonograms and \citet{butler2018framework} finds tactics for solving K-12 algebra problems, both using a CEGIS loop similar to our synthesis process. None of these address termination, although Swapper likely screens out non-terminating rulesets through its autotuning step. The Butler works both focus on synthesizing small, highly general rulesets that are similar to human rewriting strategies, unlike the Halide TRS which tolerates very large rulesets.  The \textbf{$\lambda^2$} tool~\cite{feser2015lambda} for example-guided synthesis performs inductive synthesis from examples, using a combination of inductive and deductive reasoning combined with enumerative search.  While our rewrite rules do not have the benefit of examples, it may be possible to apply this technique to obtain more sophisticated predicate synthesis for our rewrites.

Superoptimization, a process of finding a shorter or more desirable program that is semantically equivalent to a larger one, is similar to our work synthesizing right-hand side terms for candidate LHSs. STOKE~\cite{schkufza2013stochastic} uses Monte Carlo Markov Chain sampling to explore the space of x86 assembly programs, while \citet{phothilimthana2016scaling} describes a cooperative superoptimizer that searches for better programs using multiple techniques in a way that allows them to learn from each other.  Souper~\cite{sasnauskas2017souper} is a recent synthesis-based superoptimizer for LLVM, which was used in evaluating the effectiveness of Alive-Infer's{} precondition synthesis.

PSyCO~\cite{lopes2014weakest} synthesizes preconditions that guarantee a compiler optimization is semantics-preserving, using a counterexample-driven algorithm similar to our rule CEGIS loop (although not like our predicate synthesis algorithm). PSyCO finds the weakest precondition from a finite language of constraints, while the space of our predicate search is theoretically inifinite but in practice bounded by our iteration limit. PSyCO must reason about side effects by tracking read and write behavior in optimization templates, while our expression language is side effect--free. More recently, Proviso~\cite{astorga2019learning} finds preconditions for C\# programs using an active learning framework composed of a machine learning algorithm for decision trees as a black-box learner and a test generator that acts as a teacher providing counterexamples. Like this work, the logic of preconditions they synthesize is in an undecidable domain.

AProVE~\cite{giesl2004automated, giesl2017analyzing} is a tool that automatically generates proofs of termination for term rewriting systems (as well as programs in Java, C, Haskell, and so on). It employs a variety of techniques for doing so. It may prove a TRS terminations through direct termination proofs, or finding a reduction order that fits all rules in the TRS, as we do in our work, searching classes of orders including path orders, Knuth-Bendix orders, and polynomial orders. It may also prove termination through dependency pairs (finding all instances in which terms of RHSs can unify with rule LHSs, then showing that a weakly monotonic ordering holds over all dependency pairs) or by abstracting rules by their effect on term height and proving that rule application must cause term height to decrease. It also employs techniques to remove portions of the ruleset that have no effect on termination and for reducing the size of the ruleset to make termination proof search more efficient. 
A proof of termination by term height abstraction would not be useful for synthesis, since it encodes no information about progress towards a goal state. A proof of termination through dependency pairs could do so, and since it uses weakly rather than strongly monotonic orders, it could permit rewriting strategies that our technique cannot. However, this method requires reasoning over the full ruleset rather than individual rules, so using it as a specification for synthesis would result in a significantly more complicated synthesis task. However, it would be interesting to see if AProVE can find a direct termination proof for the Halide simplifier and handwritten variable solver TRSs and if those direct termination proofs are human-interpretable. If AProVE can find reduction orders to fit, it would also be interesting to see what rulesets could be synthesized using them as an input to our synthesis pipeline.

Other software that can solve equations in the manner of the Halide variable solver include SymPy~\cite{10.7717/peerj-cs.103}, MATLAB~\cite{higham2016matlab}, and GNU Octave~\cite{eaton1997gnu}. We plan to benchmark various synthesized versions of the Halide variable solver against these tools. 