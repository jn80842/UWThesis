\chapter{Verification}
\label{chapter:verification}

We improve the Halide simplifier term rewriting system by ensuring its soundness in
two ways: first, we verify that each individual rule is correct, meaning that the
rewrite preserves semantics. Then we verify that the term rewriting system is
guaranteed to terminate on all inputs by ensuring that no sequence of
rule applications, on any input expression, can form a cycle.

\subsection{Rule Verification}
\label{sec:ruleverification}
We verify each individual rule is correct by modeling Halide
expressions in SMT2 and using the SMT solver Z3~\citep{de2008z3} to
prove that the rule's left- and right-hand sides are equivalent. Most Halide expression
semantics map cleanly to SMT2 formulas. The functions \hmax{} and
\hmin{} are defined in the usual way, and \hsel{} in
Halide is equivalent to the SMT2 operator \texttt{ite}. Division and
modulo are given the Euclidean definitions in both Halide and
SMT2~\citep{boute1992euclidean}, though division and modulo by zero is handled
differently (in Halide both evaluate to zero).
%If a variable appears in the LHS of a rule as a divisor in a
%division or modulo operation, it is assumed to be non-zero. %The Halide
%expressions do not have a true boolean type (true and false are represented by
%unsigned integers of 1 bitwidth), so expressions must be typed as either
%\texttt{Int} or \texttt{Bool} when translated into SMT2. The Halide expression
Halide's TRS uses two vector-constructing operators, \texttt{broadcast} and \texttt{ramp}; all
other integer operators can be coerced to vector operators. 
\texttt{broadcast(x, l)} projects some value $x$ to a vector of length $l$; because of
the type coercion, we can simply represent \texttt{broadcast(x, l)} as the variable
\texttt{x} in SMT2. \texttt{ramp(x, s, l)} creates a vector of length $l$
whose initial entry has the value $x$ and all subsequent entries increase with
stride $s$. In SMT2, we represent this term as the symbolic expression $x + l *
s$, where $l$ must be zero or positive.

Given this modeling, for each rule, we assert any predicate guards are true, then
ask Z3 to search for a variable assignment that makes the LHS and RHS not
equivalent.  If Z3 indicates no such assignment exists, the LHS must be equivalent to
the RHS and the rule must be correct. We implemented an SMT2 printer for 
Halide rewrite rules that automatically constructs an SMT2 verification problem for each rule.
Rule verification using Z3 is fully automated
and can be run for the current set of rewrite rules used in the compiler via a script.

However, for 123
rules, Z3 either timed out or returned unknown. Nearly all of these rules used
either division or modulo. We used the proof assistant Coq to manually prove the
correctness of these remaining rules. In the course of these proofs, we
discovered we were also able to relax the predicate guards of \NumPredicatesRelaxed
rules; for example, in some cases a rule
with a guard requiring some constant to be positive would be equally valid
if the constant was non-zero.


\paragraph{Evolving Semantics}
\label{sub:evolvingsemantics}

This mostly-automated approach to verification assists with changing
the language semantics. Our initial work on verification was not on
the semantics described above: division or modulo by zero was originally
considered undefined behavior. Since we had already
modeled Halide semantics in SMT2, it was easy to alter the
definitions of division and modulo and re-run the verification scripts.
We proved \NumZdivCoqProvedRules rules manually in Coq after Z3 failed to verify them; 
since in the previous round all Coq proofs
included the assumption that all divisors were non-zero, in most cases
we had only to add a case to show that the rule was true when the
divisor was zero as well. In the course of reviewing
these proofs, we identified \NumZdivRelaxedPredicates rules whose
predicates included the condition that a divisor be non-zero and where
that condition could safely be lifted. We found that the remaining
\NumZdivFalseRules rules were not correct under the new semantics and
submitted a patch to amend them.

\subsection{Evaluation of verification work}

\begin{itemize}
 \item \textbf{Can verification contribute to the Halide TRS, or is testing alone sufficient?} Although handwritten rules have been extensively fuzz-tested and new rules are peer-reviewed before inclusion, we were still able to discover \NumRulesFixed unsound rules through verification. In addition, verification was able to identify \NumZdivFalseRules rules that needed to be changed after a significant semantics redefinition, which would have been challenging to discover through testing. (Section~\ref{sec:eval-correctness})

\end{itemize}

\subsubsection{Can Verification Contribute to the Halide TRS, or is Testing Alone Sufficient?}
\label{sec:eval-correctness}

The Halide TRS has a stringent development process: new rules are peer reviewed after they are proven on paper, and fuzzing has been discovering bugs for months. It is thus reasonable to ask whether mechanized verification can add any value. Our verification discovered \NumRulesFixed new soundness bugs and \NumPredicatesRelaxed instances of rules whose predicates were overly restrictive. The former bugs eluded the fuzzer; the latter are deemed too hard so the fuzzer does not look for them. Furthermore, because the verification infrastructure was in place, it was possible to verify a change of semantics without much additional effort, identifying 44 rules that were incorrect under the new semantics.

The first use of verification took place when the TRS had not yet been merged into the Halide master branch. We ran the verification pipeline and discovered \NumRulesFixed incorrect rewrite rules, listed in Table~\ref{tab:verfirstround}. The rules that could not be checked with Z3 were proved true using the Coq proof assistant (none of the manually proved rules were found to be incorrect). While these bugs were found automatically the fixes were performed by hand, as the synthesis pipeline did not yet exist. 

Case $\mathbb{H}$
\footnote{
\label{footnote:casesfi}
$\mathbb{F}$: \url{https://github.com/halide/Halide/pull/4721}
$\mathbb{G}$: \url{https://github.com/halide/Halide/pull/4772}
$\mathbb{H}$: \url{https://github.com/halide/Halide/pull/4439} % these are the div by 0 semantics change fixes
$\mathbb{I}$: \url{https://github.com/halide/Halide/pull/4850}
}
is a change to the semantics of Halide that may not have even been attempted without the verifier. In this change, Halide defined division or modulo by zero to evaluate to zero, instead of being undefined behavior, in response to an issue discovered by Alex Reinking~\citep{reinkingthesis}. Existing tests and real uses of Halide were useless as a test of this change, as \emph{they were all carefully written to never divide by zero}. Within the TRS, this change required rechecking every rewrite rule that involves the division and modulo operators. Whereas previously each rule assumed that a denominator on the LHS could not be zero, now it was necessary to either show that the rule was still correct in the case where a denominator was zero, or constrain the rule to only trigger when the denominator was known to be non-zero. This was done by encoding the new semantics into the verifier, and reverifying all rules. Because division and modulo is involved, these rules cannot always be mechanically verified. 
\NumZdivCoqProvedRules rules were reverified with a human in the loop by revisiting and modifying existing Coq proofs. The mechanical re-verification was all but push-button; the manual effort for updating the Coq proofs was non-trivial, but about half of the effort of writing the original proofs from scratch. In this process, \NumZdivFalseRules existing rules were found to be incorrect in the new semantics and fixed. (Two of them were in fact not related to division, but were instead the first discovery of the bugs injected in case $\mathbb{D}$ above.) The remaining 42 rules were modified to only trigger when the denominator was known to be non-zero, either by adding a predicate to the rule, or by exploiting the TRS’s ability to track constant bounds and alignment of subexpressions. Three examples of now-incorrect rules were:

\begin{align*}
(x/y)*y + (x\;\%\;y) \rewrites & x \\
  -1 / x \rewrites & \hsel(x < 0, 1, -1) \\
(x + y)/x \rewrites & y/x + 1
\end{align*}

The first was modified to:
\[
(x/c_0)*c_0 + (x\;\%\;c_0) \rewrites x \pred c_0 \neq 0
\]
and the other two were constrained to only trigger when the denominator is known to be non-zero via other means.

The cases discussed in Section~\ref{sub:bugfixes} all concern fixing existing problems while not introducing new ones. By giving a proof of soundness, showing that the ruleset is correct and that the rules are cycle-free, we also remove two entire classes of future bugs. For reference, over the life of the Halide project there have been 14 pull requests that fix incorrect rules, and 3 pull requests that modify rules in order to avoid cycles. Fixing a reduction order also guarantees that no new cycles can be introduced as long as new rules obey this order; without such a guide, it is possible to introduce a rule that would close a loop in some sequence of existing rule applications and cause a cycle, resulting in infinite recursion during compilation. 

\begin{table*}
\caption{Rules corrected through the first round of verification.}
{\renewcommand{\arraystretch}{1.2}
\begin{tabular}{r|l|l}
& Rule & Counterexample \\
\hhline{=|=|=}
Wrong &  $\frac{x * c_0}{c_1} \rewrites \frac{x}{(c_1 / c_0)} \pred c_1 \;\%\; c_0 = 0 \wedge c_1 > 0$ & $c_0 = -1, c_1 = 2, x = 1$\\
Fixed & $\frac{x * c_0}{c_1} \rewrites \frac{x}{(c_1 / c_0)} \pred c_1 \;\%\; c_0 = 0 \wedge c_0 > 0 \wedge \frac{c_1}{c_0} \neq 0$ & \\
\hhline{=|=|=}
Wrong & $(\frac{x + c_0}{c_1})*c_1 - x \rewrites x \;\%\; c_1 \pred c_1 > 0 \wedge c_0 + 1 = c_1$ & $c_0 = 2, c_1 = 3, x = -5$\\
Fixed & $(\frac{x + c_0}{c_1})*c_1 - x \rewrites {-x} \;\%\; c_1 \pred c_1 > 0 \wedge c_0 + 1 = c_1$ & \\
\hhline{=|=|=}
Wrong & $x - (\frac{x + c_0}{c_1})*c_1 \rewrites -(x \;\%\; c_1) \pred c_1 > 0 \wedge c_0 + 1 = c_1$ & $c_0 = 2, c_1 = 3, x = -5$\\
Fixed & $x - (\frac{x + c_0}{c_1})*c_1 \rewrites ((x + c_0) \;\%\; c_1) + {-c_0} \pred c_1 > 0 \wedge c_0 + 1 = c_1$ & \\

\end{tabular}
}
\label{tab:verfirstround}
\end{table*}


\section{Termination}
\label{sec:termination}

\modified{Under the umbrella goal of simplifying expressions, the Halide TRS uses
many strategies: it may attempt to make expressions as short as possible; it may factor out
vector operations or more expensive operations such as division; it may attempt to
canonicalize subexpressions so they can cancel or be shown equivalent. These
strategies are not necessarily aligned and may even undo each other. Crafting new rules 
can thus require a detailed understanding of the ruleset and its various applications. 
In this section we formalize the Halide expression simplification strategy that was
previously only encoded in the ruleset itself. In doing so, we also prove that since 
each rule strictly makes progress in accordance to this strategy, the Halide TRS always terminates.}

Consider a term
rewriting system containing only one rule: $x + y \rewrites y + x$. The term
$3 + 5$ matches the LHS of the rule and is rewritten to $5 + 3$, which can again
be matched to the rule and rewritten to $3 + 5$, and so on. Termination failures in the Halide TRS have occurred in the past\footnote{See for example https://github.com/halide/Halide/pull/1525}, causing unbounded recursion and eventually a stack overflow in the compiler. This is tricky to debug, and may not always be reported by users, since the error is fairly opaque. To show that this type of error has been eliminated, we must prove that there is no expression in the Halide expression language that can be infinitely rewritten by some sequence of rules that form a cycle.

Intuitively, we can think of Halide expressions as existing in some multi-dimensional space; when an expression is rewritten by a rule, it moves from one point in that space to another. If each rule always rewrites expressions such that they move monotonically in some direction through the expression space, then no sequence of rules can form a cycle. These directions correspond to our intuition about why certain rules are useful (like the examples at the beginning of this section). We can consider each of these directions as a dimension in the expression space. If we formalize this desirable ordering and show that all rewrites from one expression to another strictly obey it, then we will have a proof of termination.

We provide this formalism and prove that the Halide term rewriting system must terminate by constructing a \emph{reduction order}, a strict order with properties that ensure that, for an order $>$ and a rule $l \rewrites r$, if $l > r$, then for any expression $e_1$ that matches $l$ and is rewritten by $l \rewrites r$ into $e_2$, it must be true that $e_1 > e_2$. Crucially, this order is evaluated over rule terms, and not over all expressions that those terms may match. We take the definition of a reduction order and the next two theorems from~\citep{baader1999term}.

\begin{theorem}\label{theorem:terminates}
A term rewriting system $R$ terminates iff there exists a reduction order $>$ that satisfies $l > r$ for all $l \rewrites r \in R$.
\end{theorem}

A reduction order is a strict order that must be well-founded, meaning that every non-empty set has a least element with regard to the order, to prevent infinitely descending chains. It must be \emph{compatible with $\Sigma$-operations}: for all expressions $s_1, s_2$, all $n \geq 0$, and all $f \in \Sigma$:
\[
s_1 > s_2 \implies f(t_1,...t_{i-1},s_1,t_{i+1},...,t_n) > f(t_1,...t_{i-1},s_2,t_{i+1},...,t_n)
\]
for all $i, 1 \leq i \leq n$ and all expressions $t_1,...t_{i-1},t_{i+1},...,t_n$. This property means that if a rewrite rule transforms a subtree in some expression $e$, the $>$ relation is preserved between the original expression $e$ and the rewritten expression $e'$. Finally, a reduction order is \emph{closed under substitution}: for all expressions $s_1, s_2$ and all substitutions $\sigma \in \mathcal{S}ub(T(\Sigma,V))$, 
$s_1 > s_2 \implies \sigma(s_1) > \sigma(s_2)$. When we match some left-hand side term $l$ to some expression $e$, we are defining a substitution for each of the variables in $l$ with some subtree in $e$; we then use that substitution to rewrite $e$ to $e'$. If our order is closed under substitutions, we know that for any expression we match to $l$, the resulting rewritten expression will obey the ordering.

Choosing a single monotonic direction in which to rewrite expressions would be overly restrictive. 
The Halide TRS is used both to prove expressions true and to simplify them; when using it as a prover, we want to put both sides of an equality into some normal form, but it doesn't particularly matter what that form is. When using the TRS to simplify expressions, on the other hand, reducing the size of an expression has important performance benefits. Since we need an ordering that covers the full Halide simplification strategy, we make use of the following theorem:

\begin{theorem}
The lexicographic product of two terminating relations is again terminating.
\end{theorem}

Thus, our strategy in finding a reduction order to cover the handwritten ruleset is to pick an order $>_a$ such that for all rules $l \rewrites r$, either $l >_a r$ or $l =_a r$. Then, we pick another order $>_b$ such that for all rules $l \rewrites r$ where $l =_a r$, either $l >_b r$ or $l =_b r$. We continue in this way until a sequence of orders has been found such that for their product $>_{\times}$, $l >_{\times} r$ holds for the entire ruleset.  Our final ordering consists of 13 component orders.

Many of our component orders are defined using measure functions that count the number of particular operations or other features in a term. We say that $s > t$ when $s$ has more vector operations than $t$, then when $s$ has more division, modulo and multiplications operations, and so on. As a sample proof sketch of this flavor of order, consider an order $s_1 >_* s_2$ that holds when the number of multiplication operations is greater in $s_1$ than in $s_2$. We represent this through a measure function $|s_1|_*$ that returns the count of multiplication operations in $s_1$; since this function maps a term to a natural number, the order is clearly well-founded. The order is also compatible with $\Sigma$-operations; we compute our measure function as follows:


\[
|f(t_1,...,t_n)|_* = \sum_i^n |t_i|_* + \begin{cases} 1 & \textrm{if } f = * \\
                                                      0 & \textrm{otherwise}
                                        \end{cases}
\]

It clearly follows that given $|s_1|_* > |s_2|_*$, it must be true that:

\[
|f(t_1,...t_{i-1},s_1,t_{i+1},...,t_n)|_* > |f(t_1,...t_{i-1},s_2,t_{i+1},...,t_n)|_*
\]

To ensure the order is closed under substitution, we need to add one more constraint. Imagine a rule $x * 2 \rewrites x + x$. Although there are fewer $*$ symbols in the righthand term than on the left, that would not be true for a substitution $\sigma = \{x \mapsto (z * z)\}$. We add a condition that for every variable present in $s_1$, it must occur either fewer or an equal number of times in $s_2$. With this constraint there is no possible substitution that increases the value of the measure function in $s_2$ that would not result in an increase by an equivalent or greater amount in $s_1$. This gives us the order:

\[
s_1 >_* s_2 \textrm{ iff } |s_1|_* > |s_2|_* \wedge \forall x \in \mathcal{V}ar(s_1) . |s_1|_x \geq |s_2|_x
\]

Most of the component orders in the full reduction order take the form above. These orders guarantee termination no matter what sequence rewrite rules are applied to an expression. However, for part of the existing ruleset, we were obliged to take into account the order in which rules are applied in the Halide TRS algorithm.

For example, one existing rule is the canonicalization $(c_0 - x) + y \rewrites (y - x) + c_0$ where $c_0$ is a constant. If $y$ is also a constant, this rule forms a cycle with itself, and could not possibly obey any reduction order. Fortunately, the rule immediately before it in the TRS handles that specific case ($((c_0 - x) + c_1 \rewrites \texttt{fold}(c_0 + c_1) - x)$), so by this sort of non-local reasoning we know that $y$ is not a constant, and therefore the rule strictly decreases a measure which counts the number of constants on the right-hand side of an addition.

%The handwritten ruleset had many rules that eliminated the occurrence of a variable, such as $x + x \rewrites x * 2$. It seems natural to define an order based on the measure function $|s_1|_x$, but for a substitution $\sigma = \{x \mapsto 3\}$, $|3 + 3|_x \not > |3 * 2|_x$. However, the simplifier algorithm always attempts constant folding before any other rule, so we know that the rule $x + x \rewrites x * 2$ can only be invoked if $x$ is not a ground term.

%Similarly, we have several rules that factor out an occurrence of a variable and introduce the constant 0 into the expression. We define an order on the occurrences of the constant 0 by defining a measure function that takes the count of terminals or leaves in the expression and subtracts the count of the constant 0; if terms $s_1$ and $s_2$ have the same number of leaves, but more of the leaves of $s_2$ are the constant 0, then $s_1 > s_2$.

%\[
%s_1 >_0 s_2 \textrm { iff } |s_1|_{leaf} - |s_1|_0 > |s_2|_{leaf} - |s_2|_0 \wedge |s_1|_{leaf} = |s_2|_{leaf} \wedge \forall x \in \mathcal{V}ar(s_1) . |s_1|_x \geq |s_2|_x
%\]

% For the rule $\texttt{max}(x + y, x) \rewrites \texttt{max}(y, 0) + x$, the order will not hold for the substitution $\sigma = \{x \mapsto 0\}$. However, we know the rule $0 + x \rewrites x$ will be invoked before this one, so the rule cannot be evaluated on the expression $\texttt{max}(0 + y, 0)$.

Relying on non-local reasoning makes our order more brittle; if the simplifier algorithm were to be changed, the termination guarantee could be lost. However, we use only a small number of basic rules in this way, which are unlikely to be changed.

Besides giving a termination guarantee, the reduction order is necessary if we want to synthesize new rewrite rules. If we do not constrain newly-synthesized rules to obey a consistent reduction order with the existing human-written ones, they form cycles with the existing rules and cause infinite recursion in the TRS. Additionally, the reduction order is the formal encoding of the types of transformations we find desirable, so the reduction order limits synthesis to rules that rewrite expressions in a useful direction.

In constructing the reduction order, we found \NumOrderingProblems rules that contradicted a desirable ordering, and submitted patches to either delete or modify them. With this amendment, the reduction order can be shown to hold over the entire Halide ruleset, and the guarantee of termination is complete. To ensure this guarantee is preserved, we build a script that automatically checks the full set of rules in the compiler to ensure they respect the reduction order. A full description of the reduction order is given in the supplemental material.


\section{Related Work}

An \textit{equivalence graph} or egraph (introduced by \citep{nelson1980techniques}, see \citep{willsey2021egg} for a recent treatment) is a data structure used to compute applications of the rules of a term rewriting system. The algorithm builds up equivalence classes by successively applying all rules to all expressions within those classes, then queries to see if two expressions are equivalent by checking if they are present in the same class. Like our algorithm, it does not backtrack, but the egraph can require significant amounts of memory, which our algorithm avoids.

AProVE~\citep{giesl2004automated, giesl2017analyzing} is a tool that automatically generates proofs of termination for term rewriting systems (as well as programs in Java, C, Haskell, and so on). It employs a variety of techniques for doing so. It may prove a TRS terminations through direct termination proofs, or finding a reduction order that fits all rules in the TRS, as we do in our work, searching classes of orders including path orders, Knuth-Bendix orders, and polynomial orders. It may also prove termination through dependency pairs (finding all instances in which terms of RHSs can unify with rule LHSs, then showing that a weakly monotonic ordering holds over all dependency pairs) or by abstracting rules by their effect on term height and proving that rule application must cause term height to decrease. It also employs techniques to remove portions of the ruleset that have no effect on termination and for reducing the size of the ruleset to make termination proof search more efficient. 
A proof of termination by term height abstraction would not be useful for synthesis, since it encodes no information about progress towards a goal state. A proof of termination through dependency pairs could do so, and since it uses weakly rather than strongly monotonic orders, it could permit rewriting strategies that our technique cannot. However, this method requires reasoning over the full ruleset rather than individual rules, so using it as a specification for synthesis would result in a significantly more complicated synthesis task. However, it would be interesting to see if AProVE can find a direct termination proof for the Halide simplifier and handwritten variable solver TRSs and if those direct termination proofs are human-interpretable. If AProVE can find reduction orders to fit, it would also be interesting to see what rulesets could be synthesized using them as an input to our synthesis pipeline.